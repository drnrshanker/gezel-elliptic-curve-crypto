\section{Introduction\label{section-introduction}}

\IEEEPARstart{E}{ver} since Shamir's proposal \cite{shamir} in '84, there's been an interest in identity-based cryptography. Particularly Boneh and Franklin's \cite{boneh} discovery of the constructive use of pairings for identity-based encryption has helped spur on new research into possible applications and implementations.

Multitudes of protocols have seen the light, however, until recently the lack of efficient hardware accelerators for the computationally expensive pairings was always kind of a show-stopper towards implementing them. Thus most of the published implementations have a focus on speed. Implementations for area- and/or power-constrained devices were either deemed infeasible or just not interesting enough. 

In 2007 Oliveira \emph{et al.} introduced their TinyTate \cite{tinytate} implementation to the world. 2008 saw the light of TinyPBC \cite{tinypbc} and NanoECC \cite{nanoecc} from the same authors. All three papers present implementations of pairings (either the Tate or $\eta_T$) on the AT128Mega microchip of a Mica node \cite{mica}, designed for deeply embedded networks. Thus it was proven that pairings were indeed feasible for use in constrained environments, such as sensor networks.

In this paper, we will investigate the feasibility of a hardware accelerator for the Tate pairing in constrained environments. In \refsect{section-pairings} necessary parameters will be defined and we will take a look at the pairing arithmetic. \refsect{section-hardware} consist of a concise overview of the implementation's hardware. Finally, results from synthesis to an ASIC implementation will be presented along with comparisons to existing implementations in \refsect{section-results}. From these a conclusion will be drawn in \refsect{section-conclusion}.
