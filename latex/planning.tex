\documentclass[a4paper]{article}
\usepackage[dutch]{babel}   % Taal van het document (opmaakregels ed.)
\usepackage{graphicx}
\usepackage{color}

%\usepackage{url}                % Opmaak van URLs
%\usepackage[official]{eurosym}  % Euro symbool
%\usepackage{colortbl}           % Kleur tabellen

\usepackage{nonfloat}            % Voeg non-floating tables & figures toe
\usepackage{amsmath,amsfonts}    % Wiskunde

\usepackage[left=2.5cm,right=2.5cm]{geometry}	% Page layout

% Plaats van afbeeldingen
\graphicspath{{images/}}
\DeclareGraphicsExtensions{.png,.jpg}

% Zet hier de woorden waarmee de splitser problemen heeft:
\hyphenation{}

% Regel nummering van figuren
\renewcommand{\thefigure}{\thesection.\arabic{figure}}
\newcommand{\Section}[1]{\section{#1} \setcounter{figure}{0}}

% Voorkom lelijke opmaak
\clubpenalty=8000
\widowpenalty=8000
\displaywidowpenalty=8000

\hyphenpenalty=5000
\tolerance=1000

%%% Code voor figuren %%%
%\vspace{\intextsep}
%\begin{minipage}{\linewidth}
%    \begin{center}
%    \includegraphics[width=311px]{fig}
%    \figcaption{Figuur uitleg}\label{Figuur label}
%    \end{center}
%    \end{minipage}
%\vspace{\intextsep}

\title{\LARGE Planning Masterproef 2008-2009\\\Large Implementations of pairings for cryptography in constrained environments}
\author{Anthony Van Herrewege}
\date{5 november 2008}

\begin{document}

\maketitle

\section*{Doelstellingen}

Deze masterproef heeft tot doel het ontwerpen van een implementatie van pairings over elliptische krommen waarbij speciale aandacht zal besteed worden aan een compact ontwerp dat gebruikt kan worden in apparaten met beperkte rekenkracht (bv. bankkaarten en RFID tags).

Meer specifiek zal een Tate-pairing over elliptische krommen in karakteristiek 2 ge\"implementeerd worden in GEZEL. Naast het zo compact mogelijk maken van deze implementatie zal rekening gehouden worden met het feit dat een aanvaardbare rekentijd behouden moet blijven.

\section*{Planning}

\begin{description}
	\item[oktober] Literatuurstudie ECC en GEZEL, leren GEZEL. (100u)
	\item[november - december] MALU implementatie GEZEL. Structuur thesistekst uitwerken. (150u)
	\item[januari - midden maart] Pairing implementatie in GEZEL. (250u)
	\item[midden maart - midden april] Debuggen, testen, optimalisaties, trade-offs. (100u)
	\item[midden april - mei] Afwerken thesistekst en Engels artikel. (150u)
	\item[eind mei] Indienen thesistekst en Engels artikel. Presentatie maken. (50u)
\end{description}

\end{document}
