\Chapter{Conclusie \& toekomstig onderzoek}

\section{Conclusie}

Deze thesis handelde over pairings, een recente ontwikkeling op gebied van cryptografie die identiteits-gebaseerde cryptografie toelaat. Er werd onderzocht hoe de Tate pairing ge\"implementeerd kan worden in hardware. Meer specifiek werd de nadruk gelegd op een compacte implementatie die daarbovenop nog eens zo weinig mogelijk vermogen verbruikte. Een implementatie van dat type zou toegepast kunnen worden in netwerken van sensoren of toestellen met een beperkt vermogen.

Verscheidene bestaande algoritmen werden aangepast en geoptimaliseerd zodat ze met een minimum gebruik aan geheugen uitgevoerd konden worden. Er werd een geheugenontwerp voorgesteld dat een goed compromis gaf tussen grootte en verbruik. Tevens werden verschillende oppervlakte- en vermogensbesparende technieken toegepast om het uiteindelijke circuit zo goed mogelijk aan de doelstellingen te laten voldoen. Door aanpassing van de kloksnelheid en het aantal gebruikte rekenschakelingen (MALUs) kunnen de parameters van het ontwerp aangepast worden aan de individuele normen van een toepassing.

Het resulterende ontwerp is uniek in zijn soort. Het verbruik per oppervlakte (genormaliseerd voor de werkfrequentie) van het gepresenteerde ontwerp is zeer laag. Ten tijde van dit schrijven waren nog geen andere ontwerpen gepubliceerd waarbij de nadruk op compactheid en laag vermogenverbruik lag.

\section{Toekomstig onderzoek}

Als toekomstig onderzoek kan het interessant zijn te onderzoeken of het mogelijk is het ontwerp nog verder te optimaliseren. Daarbij is het waarschijnlijk nuttig te trachten de gebruikte FSM verder te vereenvoudingen, aangezien die zowat de helft van de oppervlakte van de totale implementatie inneemt.

Verder zou ook onderzocht moeten worden in welke mate het werken over een groter Galois veld de oppervlakte en het verbruik wijzigt. Ook de implementatie van andere types pairings kan interessant zijn. Zo bestaan er bijvoorbeeld de $\eta_T$ en de modified Tate pairing die beiden berekend kunnen worden in ongeveer de helft van de tijd nodig voor de berekening van de Tate pairing.
